\documentclass{article}
\usepackage[utf8, margin=1.3in, top=1in, bottom=1in]{geometry}

\title{Meeting Summons}
\author{Mikael Falkenberg Krog \and Magnus Gluppe \and Jakob Frantzvåg Karlsmoen \and Mikkel Aas}
\date{February - April 2020}

\usepackage{natbib}
\usepackage{graphicx}

\begin{document}

\maketitle
\tableofcontents
\newpage
\section{Meeting summons}
\subsection[Meeting summon: Project Group 2]{Meeting summon: Project Group 2\\ {\large 28.02.2020, 11:00, Atriet A-bygget NTNU Gjøvik}}

Magnus Gluppe is the meeting Chairman and minutes responsible for this meeting.
\newline
\newline
\large Agenda 
\begin{itemize}
    \item Case no. 01/2020:  Opening meeting
    \item Case no. 02/2020:  Approval of agenda
    \item Case no. 03/2020:  Eating protocol during meetings
    \item Case no. 04/2020:  Startpoint and endpoint for a work sessions
    \item Case no. 05/2020:  Group versus individual work
    \item Case no. 06/2020:  Briefing everyone on individual work
\end{itemize}
\newline
\newline
Please contact me if you are unable to attend the meeting.
\newpage

\subsection[Meeting summon: Project Group 2]{Meeting summon: Project Group 2\\ {\large 19.03.2020, 19:30, Online Call}}

Mikael is the meeting Chairman and minutes responsible for this meeting
\newline
\newline
\large Agenda 
\begin{itemize}
    \item Case no. 01/2020:  Opening meeting
    \item Case no. 02/2020:  Approval of agenda
    \item Case no. 03/2020:  Compression?
    \item Case no. 04/2020:  Workings of export
    \item Case no. 05/2020:  Adding folders instead of individual images
    \item Case no. 06/2020:  Consistency in UI
    \item Case no. 07/2020:  Communication
\end{itemize}
\newline
\newline
Please contact me if you are unable to attend the meeting.
\newpage

\subsection[Meeting summon: Project Group 2]{Meeting summon: Project Group 2\\ {\large 03.04.2020, 12:00, Online Call}}

Mikkel is the meeting Chairman and minutes responsible for this meeting
\newline
\newline
\large Agenda 
\begin{itemize}
    \item Case no. 01/2020:  Opening meeting
    \item Case no. 02/2020:  Approval of agenda
    \item Case no. 03/2020:  Delegate report work
    \item Case no. 04/2020:  Optimization
    \item Case no. 05/2020:  Write JUnit tests
    \item Case no. 06/2020:  The edit image buttons
    \item Case no. 07/2020:  Connect to database
\end{itemize}
\newline
\newline
Please contact me if you are unable to attend the meeting.
\newpage

\subsection[Meeting summon: Project Group 2]{Meeting summon: Project Group 2\\ {\large 22.04.2020, 12:30, Online Call}}

Jakob is the meeting Chairman and minutes responsible for this meeting
\newline
\newline
\large Agenda 
\begin{itemize}
    \item Case no. 01/2020:  Opening meeting
    \item Case no. 02/2020:  Approval of agenda
    \item Case no. 03/2020:  What is left to do on the application
    \item Case no. 04/2020:  What is left to do on the report
    \item Case no. 05/2020:  How has the project been?
\end{itemize}
\newline
\newline
Please contact me if you are unable to attend the meeting.


%%%%%%%%%%%
% MINUTES %
%%%%%%%%%%%


\newpage
\section{Minute}
\subsection{Minute from project meeting in Project Group 2}
\newline
\textbf{Time/location:} 28.02.2020, 11:00, Atriet A-bygget NTNU Gjøvik
\newline
\textbf{Present: }Mikkel Aas, Magnus Gluppe, Jakob Karlsmoen, Mikael Krog
\newline
\textbf{Absent:} No one
\newline
\textbf{Moderator:} Magnus Gluppe
 \newline \newline
\textbf{Case no 2/2020} \newline
Approved by acclamation.  \newline  \newline
\textbf{Case no 3/2020}  \newline
Anyone can eat during the meeting or work sessions, but we should avoid noisy food like chips. At important deadlines, the group can go out to eat. \newline  \newline
\textbf{Case no 4/2020}  \newline
Issue undecided, we like the approach of working until we are done with what we started. However, this method can have its disadvantages. Not everyone is productive at the same time, and might wish to end a session early and pick up the work later. If no one else does this, you can feel obligated to stay and produce a sub-par product.  \newline  \newline
\textbf{Case no 5/2020}  \newline
When we are further along with the project, it will become easier to work individually. The group still wishes to have joint work sessions, even if our tasks do not overlap. This ensures a certain amount of structure for everyone, and we can ask each other for help.  \newline  \newline
\textbf{Case no 6/2020}  \newline
In this part of the project, all the group members have cooperated on various documents and diagrams. So this point is not very relevant to this stage of the project. 

\begin{flushright}28.02.2020, Magnus Gluppe\end{flushright}

\newpage
\subsection{Minute from project meeting in Project Group 2}
\newline
\textbf{Time/location:} 19.03.2020, 19:30, Online Call
\newline
\textbf{Present: }Mikkel Aas, Magnus Gluppe, Jakob Karlsmoen, Mikael Krog
\newline
\textbf{Absent:} No one
\newline
\textbf{Moderator:} Mikael Krog
\newline \newline
\textbf{Case no 2/2020} \newline
Approved by acclamation.
\newline  \newline
\textbf{Case no 3/2020}  \newline
We have decided that compression is not to be prioritized, and instead try to find other solutions. We may come back to compression later.
\newline  \newline
\textbf{Case no 4/2020}  \newline
The export function is currently under development.
\newline  \newline
\textbf{Case no 5/2020}  \newline
We have some ideas to import multiple images is imported at once. We will begin testing this later.
\newline  \newline
\textbf{Case no 6/2020}  \newline
It will be taken into consideration.
\newline  \newline
\textbf{Case no 7/2020}  \newline
We will keep on using discord as communication, it is working well.
\begin{flushright}19.03.2020, Mikael Krog\end{flushright}


\newpage
\subsection{Minute from project meeting in Project Group 2}
\newline
\textbf{Time/location:} 03.04.2020, 12:00, Online Call
\newline
\textbf{Present: }Mikkel Aas, Magnus Gluppe, Jakob Karlsmoen, Mikael Krog
\newline
\textbf{Absent:} No one
\newline
\textbf{Moderator:} Mikkel Aas
\newline \newline
\textbf{Case no 2/2020} \newline
Approved by acclamation.
\newline  \newline
\textbf{Case no 3/2020}  \newline
We have chosen to write different segments that will be delegated evenly to the group members. 
\newline  \newline
\textbf{Case no 4/2020}  \newline
The program is using too much RAM. There is still a few things we can try to fix this. We can try to Make scenes static. This ensures that one scene has one controller and not several controllers, which might save RAM.
\newline  \newline
\textbf{Case no 5/2020}  \newline
We have to write JUnit tests to ensure that every section of the application meets its design and behaves as intended.
\newline  \newline
\textbf{Case no 6/2020}  \newline
We should combine the two edit buttons into one. This makes the GUI more user friendly and removes clutter.
\newline  \newline
\textbf{Case no 7/2020}  \newline
We have to move our database from an embedded database within the application to a remote database on the NTNU servers. This will be done with simple configurations in our persistence.xml document.
\begin{flushright} 03.04.2020, Mikkel Aas\end{flushright}

\newpage
\subsection{Minute from project meeting in Project Group 2}
\newline
\textbf{Time/location:} 22.05.2020, 12:30, Online Call
\newline
\textbf{Present: }Mikkel Aas, Magnus Gluppe, Jakob Karlsmoen, Mikael Krog
\newline
\textbf{Absent:} No one
\newline
\textbf{Moderator:} Jakob
\newline \newline
\textbf{Case no 2/2020} \newline
Approved by acclamation.
\newline  \newline
\textbf{Case no 3/2020}  \newline
Need small cosmetic fixes. There are also other small changes that could be made, but are not important so they are not a priority. Compile to jar? Make sure JavaDoc is good and consistent. 
\newline  \newline
\textbf{Case no 4/2020}  \newline
Need to flesh out "how the assignment was solved" and "overview". "Further work" needs a lot of work. Some parts need a little padding. Conclusion needs to be written. 
\newline  \newline
\textbf{Case no 5/2020}  \newline
Probably should have started writing the report earlier. There was a lack of interest in the subject matter, therefore we did not give it the attention it deserves and could have been better at following it more closely during the project. 
\begin{flushright} 22.04.2020, Jakob Karlsmoen\end{flushright}
\end{document}
