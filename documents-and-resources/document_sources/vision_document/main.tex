\documentclass{article}
\usepackage[utf8, margin=1.3in, top=1in, bottom=1in]{geometry}
\usepackage{tabularx}
\usepackage[table]{xcolor}

\title{Vision Document}
\author{Mikkel Aas \and Magnus Gluppe \and Jakob Frantzvåg Karlsmoen \and Mikael Falkenberg Krog}
\date{February 2020}

\begin{document}

\maketitle
\section*{Revision History}
\begin{tabularx}{1.0\textwidth} { 
  | >{\raggedright\arraybackslash}X 
  | >{\raggedright\arraybackslash}X
  | >{\raggedright\arraybackslash}X
  | >{\raggedright\arraybackslash}X | }
    \hline
    \rowcolor{lightgray} Date & Version & Description & Author \\
    \hline
    26.02.2020  & 0.1  & First revision & Mikkel Aas, Magnus Gluppe, Jakob Karlsmoen, Mikael Krog   \\
    \hline
    27.02.2020  & 0.2  & Second revision & Mikkel Aas, Magnus Gluppe, Jakob Karlsmoen, Mikael Krog   \\
    \hline
    28.02.2020 & 0.3 & 1st iteration submission & Mikkel Aas, Magnus Gluppe, Jakob Karlsmoen, Mikael Krog \\
    \hline
    19.03.2020 & 0.4 & 2nd iteration submission & Magnus Gluppe, Mikael Krog \\
 	\hline
 	21.04.2020 & 0.5 & Responding to feedback & Magnus Gluppe, Mikael Krog \\
 	\hline
\end{tabularx}
\newpage
\tableofcontents
\newpage

\section{Introduction}
This document outlines the vision of MetIma, an image application. MetIma is a newly founded development team that has been subcontracted by Evil Inc. to create an image application. The goal of the project is to design and develop an application that gathers metadata and organizes images. Evil Inc. has large amounts of image data from their users, and need a convenient way to organize and store them. We aim to deliver a product that is easy to use, while still being sophisticated. User interaction and design are key values in our development team. The product should be designed with good programming principles in mind, and use modern programming philosophy. This will ensure that the MetIma application is not rapidly outdated and stays relevant in the market.

\subsection{Purpose}
This vision document outlines the the purposes and aspirations for our image application MetIma. Here we state what we wish to achieve and how we want to achieve it.
\subsection{Scope}
The MetIma application can have potential use for both  consumers and private businesses. However, the scope of this project is to creating a functioning and well design application. We want MetIma to be usable on a wide variety of devices and operating systems. However, the development team is limited to four relatively inexperienced programmers and a timetable of 3 months, so the end product cannot expect a substantial market share or high levels of sophistication and polish.
\subsection{Definitions, Acronyms and Abbreviations}
\begin{itemize}
    \item SQL - Structured Query Language
    \item NTNU - Norwegian University of Science of technology
    \item NOK - Norwegian Krone
    \item ORM - Object-Relational Mapping
    \item Inc. - Incorporated

\end{itemize}{}
\subsection{Overview}
The rest of the vision document describes what goals we have for this project, and how we intend to achieve them. It also contains risk analysis and cost estimations. The end product we aim for is described in the vision document, and the different requirements we have are included.

\section{Positioning}
\subsection{Problem Statement}
Digitally stored images are the norm today, and can be a struggle to sort and organize. When the images are stored in a file system, there is no easy native way to organize these images unless the user does it manually. This makes it cumbersome to store images digitally. This exact problem affects a lot of users. A solution to this problem is an application that can take these images and organize them automatically for the user, which is what MetIma will be able to do. 
\subsection{Product Statement}
This product is made a corporation who had a demand for organizing their image data, but can be of use to any individual that stores images and photos digitally. Today, that covers almost every person in the developed world. Many of these individuals are in need of a way to organize these files, and that is where our application comes in. Evil Inc. currently owns the license for MetIma, but the application can be of interest for other businesses in the future. The application is connected to a database, but processing happens locally on your computer,

\section{Project goals}

\subsection{Efficiency goal}
We want our application to enable an easier way to organize digital image collections. By giving the user an automated way to have their images organized, it will be easier for them to work with their collection and find the images they need when they want them. The application will be easy to use for all users no matter their computer proficiency.

\subsection{Result goals}
The main result goal for this application is to increase productivity for the user. The user in this case is Evil Inc., a private business. Both for storing and accessing the digitally stored images. The users productivity is greatly dependent on the users technical proficiency. However, an indication of a reached goal is a significant speed increase in the time it takes for a user to organize a set of images and later to access a specific image.

The other result goal is increased earnings in a business environment. While this also partially included in the increased productivity goal, this goal is more of a commercial goal. In a world where "time is money", the less effort a user has to put in to doing mundane tasks, the better. A user can spend more time with their clients, or have the resources to bring in more clients. Increasing the number of clients, results in higher revenue.

\subsection{Process goals}
During the process we want to improve the developers competence in coding and development. By learning to use different libraries, tools and strategies the developers will improve their personal skills. They will be able to apply this newfound competence in future projects, and also in this project by refactoring code at a later stage in the development process. 

Taking into consideration that this is a student project, there are a lot of positive learning experiences that can be obtained. We have to work closely with other students, getting to know them better and creating a functionally working environment. There are several new concepts in the realm of documentation and project planning, this can be vital for future projects in a business setting.

\section{User description and Stakeholders}
\subsection{Market Demographics}
MetIma aspires to reach a vast audience for our product, by using open source software and cutting edge technology. We strive to have the MetIma application work cross platform on all devices. This would make MetIma a very flexible product, anyone with a smartphone or computer can enjoy our product. However, as our company is a currently unknown startup, the initial launch will be on a smaller scale.
\subsection{Stakeholders}
The stakeholders of the project is the development team themselves and Evil Inc. The success of the application is vital for all the developers future careers. Evil Inc.'s stakes in the application is the daily operations of their business and   \vspace{0.2in} \newline
\begin{tabularx}{1.0\textwidth} { 
  | >{\raggedright\arraybackslash}X 
  | >{\raggedright\arraybackslash}X
  | >{\raggedright\arraybackslash}X | }
    \hline
    \rowcolor{lightgray} Name & Description & Responsibilities \\
    \hline
    Developer & The developer is in charge of developing and maintaining the application &
    \begin{itemize}
        \item[--] Developing the product
        \item[--] Maintaining the application 
        \item[--] Meet the user requirements
        \item[--] Ensure the viability in the marketplace
    \end{itemize}{} \\
    \hline
     Client & This is the person or organization that has commissioned this project &
    \begin{itemize}
        \item[--]  Paying the developers.
        \item[--]  Give specific requirements for the product.
        \item[--]  Using the application in a responsible way.
    \end{itemize}{} \\
    \hline

\end{tabularx}

\subsection{Users}
\begin{tabularx}{1.0\textwidth} { 
  | >{\raggedright\arraybackslash}X 
  | >{\raggedright\arraybackslash}X
  | >{\raggedright\arraybackslash}X | }
    \hline
    \rowcolor{lightgray} Name & Description & Responsibilities \\
    \hline
    User & End user. The employees that will be using the application in their work & \begin{itemize}
        \item[--] Getting familiar with the application, to be able to utilize it in an effective manner.
        \item[--] Provide useful feedback for future improvements.
        \item[--] Use the platform regularly so it becomes part of their routine and workflow.
        \item[--] Maintaining security so no sensitive information is compromised.
    \end{itemize}{} \\
    \hline
\end{tabularx}


\section{Product Overview}
\subsection{Product Perspective}
Our product is independent and self-contained. It does not belong to a larger system. This product is an alternative to other photo gallery solutions. 
\subsection{Risk Analysis}
There are multiple risks which could be detrimental to the project: 


\begin{tabularx}{1.0\textwidth} {
  | >{\raggedright\arraybackslash}X
  | >{\raggedright\arraybackslash}X | }
    \hline
    \rowcolor{lightgray} Risk & Risk Probability \\
    \hline
    The team could overreach, and try to develop something unachievable, which would make our end product unusable. & unlikely \\
    \hline
    The developers may not be able to exclusively work on this project, this might cause delays. & Very likely \\
    \hline
    There is a risk of exceeding our budget. & unlikely \\
    \hline
    Not being able to finish the product before the deadline. & unlikely \\
    \hline
\end{tabularx}


\subsection{Estimated costs}
The development costs of the project is all our developers salary. Our developers hourly salary rate is 1470 NOK.
Each team member is estimated to spend 150 hours on the project. +/- 10\%.
We have four developers working on this project. This means our expected development cost is a minimum of about 800 000 NOK and a maximum of about 970 000 NOK. 
Location and development tools are supplied by NTNU. Therefore, we have no other estimated costs.


\section{Product Features}
This section will present the first draft of the functional features of the MetIma application.

\subsection{The home page}
The first page the user is represented with is the home page. The homepage contains three buttons and the title of the application. Since the two main features of the application are to view a gallery and add new images, we thought it would be natural to have these two options as buttons on the homepage. The third button on the home page toggles dark mode for the application.

\subsection{The gallery page}
The gallery page mainly consists of an area where all the pictures in the gallery are presented to the user. The second biggest feature of the gallery page is the search bar. The search bar allows the user to search for images based on tags, filenames, and metadata. The user will be able to export the search results to a PDF document with an export button. There will also be home button and an add new image button that grants the user quick access to their respective pages.

\subsection{The add new image page}
 To add a new image the user either has to click the button on the home page or use one of quick access buttons that are placed in the top left corner. When the user clicks the button they will be prompted to choose either a singular image, or a folder of images on the computer. If the user chooses a singular image they will be taken to a page where they can set the filename and add appropriate tags. However, if the user chooses a folder, all the pictures will have the filename followed by a sequence of numbers, and they will all share the same tags. Before the image or images are added to the program, the user will have to confirm or deny the import by clicking either add or cancel. The page will contain relevant quick access buttons in the upper left corner. 

\subsection{The image-view page}
The last and simplest page is the view image page. It encompasses two main features. The first is the ability to view the selected image. The second is the availability to view a list of the metadata of the image, which you can simply scroll through. This page will also contain quick access buttons.

\section{Product Requirements}
For this project we have a multitude of requirements. Including System requirements, technical requirements and documentation requirements. 
\subsection{System Requirement}
The only real system requirement for our application is to have Java Runtime Environment installed on your computer. The application is written in Java 11, and therefore Java will be needed to run the program.

\subsection{Technical Requirements}
The technical requirements describes which functions the application has to have.
The technical requirements for the application are as follows:
\begin{enumerate}
    \item The application must be made as a standalone Java application.
    \item Use the MySQL database at the university and use ORM technology the way it has been taught on the programming course.
    \item Use a connection  pool  with  one  connection. This is to improve performance and avoid using too many database connections.
    \item Test at least all the classes that read image metadata with JUnit.
    \item The application has to follow Web Content Accessibility Guidelines 2.1 Perceivable.
\end{enumerate}

\subsection{Documentation Requirements}
The project has multiple documentation requirements. These define what we must document during the development:
\begin{enumerate}
    \item The project has to be carried out in three iterations. 
    \item The team must use collaboration tools as part of the project. Such as GitLab, Google Drive etc.
    \item The team must submit a main report which includes evaluation of the teamwork, experiences dealing with the project-work and cooperation within the team.
    \item The main report must contain the following attachments: collaboration agreement, Gantt chart, time-sheet, meeting invitations and minutes, vision document, link to GitLab WIKI pages and a link to JavaDoc on GitLab pages
\end{enumerate}
\end{document}