\section{How the assignment was solved}
\subsection{Use of Literature and Internet}
A programmer has a variety of different tools at their disposal. The most used website has been Stack Overflow. It is a discussion forum where you can post code related questions and have other programmers answer and provide examples. \cite{website:Stackoverflow} To ensure this project goes smoothly the development team was sent to a Software Engineering course. For the projects structure, vision document and the report, the team has used the resources made available to us through this course. This includes the lectures we attended and the multitude of templates provided by the lecturer for this course. 

\subsection{Project Responsibilities}
The work on the project was loosely divided between the four team members. Each team member had an area of responsibility: Mikael was the group leader, Magnus had the responsibility of organizing meetings, Jakob was responsible for the documentation and Mikkel was head of design. However, when it came to developing the application itself, the roles were more fluid. Everyone collaborated and worked in a group, what you worked on was mostly decided in that work session. The one constant was that Mikkel has focused on the design aspects of the project.

\subsection{Methods and Standards Used}
Due to the structure of the project we chose to use Scrum. This is a very agile model with a high  amount of customer involvement. This was done by having several partial deliveries of the application for review. While the requirements of the project were not changed, this provided us with valuable feedback on the progress of the application. This made it possible for the development team to make adjustments and focus on working software. 

For organizing the project we used the Unified Process, or UP. The Unified Process is structured in a series of faces, each which is complete when you reach a milestone. An example of this is when the MVP was done, or when the application was connected to a database. Phases can overlap, and often do, as not all group members worked on the same task at the same time.

\subsection{Brief description of standard software used}
\begin{itemize}
    \item Java 11.0
    
    The entire project was written in Java 11.0. Java is a class-based and object oriented general purpose programming language. The language is designed to have as few implementation dependencies as possible, and it runs on every platform that support Java. \cite{website:JavaWikipedia}
    \item IntelliJ IDEA
    
    The IDE of choice was IntelliJ and all group members used this development tool. IntelliJ IDEA is an integrated development environment for developing computer software. It is developed by JetBrains and written in Java. It features tools like coding assistance, that improve efficiency while coding. \cite{website:Jetbrains}
    \item GitLab 
    
    The collaboration was done using GitLab, where we all shared a project. GitLab is a web-based DevOps life-cycle tool that provides a Git-repository manager. The Git-repository manager has features such as issue tracking, and continuous integration and deployment. \cite{website:Gitlab} 
    \item Javafx Scene Builder
    
    For GUI we opted to use Javafx Scene Builder instead of writing fxml documents from scratch. Javafx Scene builder is a software that lets the user create fxml files by dragging and dropping controls from a palette.
    \item BalsamiQ Wireframes 
    
    When creating our MVP we based the design on the previously made wireframe. The wireframe was made using a software called BalsamiQ Wireframes. BalsamiQ Wireframes is a graphical user interface website wireframe builder application. It contains pre-built widgets that the user can arrange using a drag-and-drop editor. We chose to use the Google Drive version of the application because it was the easiest to share.\cite{website:SceneBuilder}
    
    \item Overleaf
    
    We used overleaf to write all the documents. Overleaf is an online Latex editor. It is a modern collaborative authoring tool that helps make science and research faster, more open and more transparent. \cite{website:Overleaf}
\end{itemize}{}