\section{Further Work}

\subsection{Product Perspective}
\subsubsection{Necessity}
The need for the product was recognized when our client contracted us to develop it. We decided to provide the product as the scope seemed manageable based on our previous experience. The scope of the product was an imaging application that could read image metadata and sort images based on tags, filenames and metadata. It was going to be a GUI based desktop application with a database connection. We gathered the requirements for the product based on the scope and began inventing.

\subsubsection{Invention}
The product was designed based on the requirements. We made a domain model that covered the functional aspects of the program and a Wireframe for the GUI layout. The application was then created based on its design. We made tests to ensure that the program worked as intended under the development phase. However, the tests were mainly focused on the data handling. 

\subsubsection{Release}
The product is going to be packaged for outside release as a jar file. The product will go through further external testing which will be done by Evil inc. The product will be handled off to an operational.

\subsubsection{Support}
The product may be supported by a new operational group and a user service group. And the code may be developed further and kept up to date be these new groups.

\subsubsection{Desupport}
Since this product is in a very competitive field, there may be no need for this specific application. Therefore there is a chance the product will be phased out in favor of better applications made by larger corporations. 


\subsection{Product Functions}
This product is an imaging application which saves image metadata. It provides functions for the user to save images and view their metadata.

The main features in the MetIma application is as follows:
\begin{itemize}
    \item Gallery: shows all the images which are added in a gallery.
    \item PDF exporting: exports all images currently shown in the gallery.
    \item Metadata reading: the application reads the image metadata and stores it in a database. The metadata can be shown when the user clicks on an image.
    \item Search: the gallery can be searched based on tags, filename and metadata.
    \item Add images: the user can add multiple images.
    \item Remove images: images can be removed from the gallery.
    \item Edit images: the tags and filenames of the images are editable.
\end{itemize}

\subsection{Assumptions and Dependencies}

\subsubsection{Assumptions}
The application assumes that the user has some basic knowledge operating a computer. We assume the user has an interest in viewing meta data, this is a feature that is more useful for photographers and similar professions. 

\subsubsection{Dependencies}
\begin{itemize}
    \item This application relies on Java and JavaFX. The user needs to have java runtime environment installed on their computer
        \item The application also relies on the user being connected to the internet, this is to save the data to a database.
\end{itemize}