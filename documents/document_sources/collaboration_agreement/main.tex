\documentclass{article}
\usepackage[utf8, margin=1.3in, top=1in, bottom=1in]{geometry}

\title{Collaboration Agreement for Gruppe 2}
\author{Mikkel Aas, Magnus Gluppe, Jakob Karlsmoen, Mikael Krog}
\date{February 2020}

\begin{document}

\maketitle

\section{Introduction}
The Collaboration agreement is based on a collection of goals, role responsibilities, procedures and guidelines for interaction within the team. The agreement is developed and complemented by the team members with their own interpretations of what you mean by these and how to achieve the goals.

\section{Goals}
\subsection{Effect Goals}
\begin{enumerate}
    \item To become acquainted, build trust in each other and increase motivation for the project there will be held regular study sessions, where the members will work as a team and help each other if any problems should arise. There will also be social gatherings to get more comfortable with each other. 
    \item Respect each other's opinions and suggestions.
    \item Finish the individual assigned work, as to not slow down and / or cause problems for others work.
\end{enumerate}
   
\subsection{Result goal}
    The group should be able to hand in all deliveries on schedule. To achieve this, all members will have a clean overview of everyone's work, and the progress of each individual. Members can then see who has done what, and what remains to be done. There is also the option to assign more people to tasks if deemed necessary.
    
\section{Roles and responsibilities}
\begin{itemize}
    \item \textbf{Project leader} \newline Krog was elected project leader by acclamation. His responsibilities involve but are not limited to, delegating assignments, keeping the schedule, solving potential conflicts and the overall vision of the project.
    \item \textbf{Organization responsible} \newline Gluppe has been given the assignment of organising the teams meetings. He will be responsible for leading and documenting the meetings during the project. Considering the groups familiarity with each other, the summons will be quite informal, but still enforced.
    \newpage
    \item \textbf{Documentation responsible} \newline All group members are responsible for documenting their work, both their work hours and what work they have done on the current task through GitLab. However, Karlsmoen has the responsibility of coordinating all the teams documentation into a cohesive, easily comprehensible document.  
    \item \textbf{Head of Design} \newline The UI/UX design is an important part of the application. Aas has the responsibility of making sure the UI/UX is up to standard, and meets all requirements. While other members will work on the UI/UX as well, everything must be approved by the head of design if it is to be included in the final design.
\end{itemize}

\section{Procedures for the teamwork}
\begin{itemize}
    \item \textbf{Meetings} \newline Meetings are to occur at least once a week. The organization of meetings is handled by Gluppe, therefore he summons the group to meetings. The idea is one formal meeting a week, with additional working sessions. This number can be increased if deemed necessary.
    \item \textbf{Notification in case of absence or other incidents} \newline If a member is late for a meeting or cannot attend, the member must contact the one responsible for organising the meeting. Either contact them on Discord or by phone. The person that is late for the meeting is still allowed to attend, unless it is a repeated problem, which will require more attention to solve. 
    \item \textbf{Documents} \newline The main method for sharing documents is through Google Drive and GitLab. Google Drive will mainly be used for documents such as this. The main report for the project will be written in Latex using Overleaf. GitLab will be the main resource for sharing code and documentation. Both manual documentation of the code, and auto-generated JavaDoc is going to be handled on GitLab.
    \item \textbf{Policy for monitoring tasks} \newline All tasks should be listed as issues on the GitLab project. The issues will be placed on a Kanban board, and updated according to the progress being made on them. This way all members will have a good overview of how the project is coming along. 
    \item \textbf{Submission of teamwork} \newline To ensure the quality of all members' work there will be a great use of GitLab commit logs to inspect the quality of members’ work. This way there will be more than one pair of eyes on anything added to the project. All group members must coordinate with Karlsmoen to ensure proper documentation. 
\end{itemize}
\section{Interaction}
\begin{itemize}
    \item \textbf{Attendance and preparation} \newline For meetings, lectures and work sessions there will be a specified time to meet. To make room for errors out of members control, the session  will start 15 minutes after the agreed upon meeting time. During the 15 minutes between the set meetup time and actual start, it is encouraged for members to prepare themselves. Any follow-up meetings will be decided upon during the meeting, and follow the same structure as a normal meeting. 
    \newpage
    \item \textbf{Presence and commitment} \newline During collective work sessions, it is expected that everyone focuses on actual work, and takes dedicated breaks when needed. If a person decides to take an individual break, the person should not disturb the other still working members, and preferably should leave the working area. 
    \item \textbf{How to support each other} \newline During meetings, there will be a summary of the work that has occured since the last meeting. Every member then has the opportunity to show off what they have done and receive feedback. This allows for everyone to have a good overview of how the project is coming along, and to help each other if someone is having trouble with their tasks. 
    \item \textbf{Disagreement, breach of contract} \newline Considering the previous work experience and familiarity of the group, there is not expected any major conflicts. However, if one arises, the involved group members will try to solve it themselves. If this turns out to be hard, the problem will be tackled by the group leader. If this does not help either, there will be brought in a mediator from outside the group, possibly from the school. If the breach of contract is critical enough, the team member will be removed from the project.
\end{itemize}

\vfill
\section{Signatures}
\vspace{25px}
\noindent\begin{tabular}{ll}
\makebox[2.5in]{\hrulefill} & \makebox[2.5in]{\hrulefill}\\
Project Leader & Date\\[8ex]
\makebox[2.5in]{\hrulefill} & \makebox[2.5in]{\hrulefill}\\
Organization Responsible & Date\\[8ex]
\makebox[2.5in]{\hrulefill} & \makebox[2.5in]{\hrulefill}\\
Documentation Responsible & Date\\[8ex]
\makebox[2.5in]{\hrulefill} & \makebox[2.5in]{\hrulefill}\\
Head of Design & Date\\
\end{tabular}
\vspace{25px}

\end{document}
